\documentclass[a4paper]{article}
\input{config}

\title{title}
\def\major{major}
\def\class{classid}
\author{name}
\def\studentId{id}

\date{\today}

\begin{document}

\maketitle
\begin{spacing}{1.2}%魔法参数,行间距
    \tableofcontents
\end{spacing}
\thispagestyle{main}

\clearpage

\setcounter{page}{1}
\renewcommand{\thepage}{\arabic{page}}
\section{需求分析}\seccontent

\sectionbreak
\section{系统总框图}

\begin{figure}[htbp]
    \centering
    \includegraphics[width=\textwidth,height=8cm]{fig/struct.png}
    \caption{系统结构图}
\end{figure}

\begin{figure}[htbp]
    \centering
    \includegraphics[width=\textwidth,height=8cm]{fig/StructPic.png}
    \caption{系统流程图}
\end{figure}

\sectionbreak
\section{各模块的设计分析}
\sectionbreak
\section{定义的函数及说明}

\sectionbreak
\section{存在的问题与不足及对策}
\subsection*{对策}

\sectionbreak
\section{使用说明}
\begin{description}
    \seccontent
    \item[程序信息]
        作者:\href{https://www.cnblogs.com/Lovechan/}{魏程浩},
        \href{https://github.com/Cuber-Wei}{@Cuber-Wei}.

        版本信息:2023/3/23, v1.0。

    \item[系统简介]
        此系统为计算题训练系统,目的在于辅助学生进行计算题练习,
        提高计算能力。

\end{description}

\sectionbreak
%  附录  %
\clearpage
\appendix
\phantomsection
\addcontentsline{toc}{section}{附录(源码展示)}
\section*{附录(源码展示)}
% 重置附录 section 格式
\renewcommand{\thesubsection}{\thesection\Alph{subsection}}

\subsection{主程序源代码(main.cpp)}


\end{document}